\subsection{基础设施}
基础设施是指Serverless底层的运行基础,
需要支持虚拟化/资源隔离/扩缩容等,
通常有几种选择:

\begin{itemize}
    \item \textbf{基于Docker}:基于Docker平台加上自主开发的框架实现,以及基于Docker的FaaS平台(如\textbf{Fn Project})
    \item \textbf{基于Kubernetes}:Kubernetes已经是事实上的云原生标准,因此使用Kubernetes作为底层运行时是一个最常规的选择,且其本身自带了容器编排、服务治理等功能,生态较好
    \item \textbf{基于Knative}:谷歌开源的基于Kubernetes的Serverless运行环境
    \item \textbf{Firecracker/microVM}:亚马逊开源的虚拟化技术,并用在AWS Lambda、Fargate等产品上
    \item \textbf{Apache OpenWhisk}: 一个开源的FaaS平台,支持在Kubernetes、Docker等多种平台
    \item \textbf{EasyFaaS}: 百度开源的函数计算服务引擎,支持多种运行平台,包括 Docker、Kubernetes 及裸机等
\end{itemize}

部分产品的架构选型如\cref{table:serverless-infras}所示。

\begin{table}[h!]
\centering
\begin{threeparttable}
    \begin{tabularx}{\textwidth}{|l|X|}
        \toprule
        \textbf{架构} & \textbf{产品} \\
        \midrule
        Kubernetes & 美团Nest\cite{meituan_serverless_nest, meituan_serverless_2}、ByteFaaS\cite{bytedance_faas}、Nimbella\cite{nimbella_k8s} \\
        \hline
        Docker & 阿里函数计算\cite{aliyun_faas_arch_2}\tnote{1}、Oracle Functions\cite{overview_oracle_functions} \\
        \hline
        Knative & Google Cloud Run\cite{gcr_knative}、工商银行Serverless\cite{icbc_faas_arch}\tnote{1}  \\
        \hline
        Firecracker/microVM & AWS Lambda、AWS Fargate\cite{firecracker_home}、腾讯云函数\cite{tecent_faas_cold_start, tecent_serverless}\tnote{2} \\
        \hline
        Apache OpenWhisk & IBM Cloud Functions\cite{how_ibm_cloud_functions_works} \\
        \hline
        EasyFaaS & 百度函数计算CFC\cite{baidu_serverless_arch}、工商银行Serverless\cite{icbc_faas_arch} \\
        \bottomrule
    \end{tabularx}
\begin{tablenotes}
    \item[1] 阿里函数计算第二代架构已经改为用神龙裸金属+安全容器实现\cite{aliyun_faas_arch_2}
    \item[2] 腾讯云使用自研的轻量级虚拟化技术,不确定是否与microVM是相同的技术
    \item[3] 工商银行函数计算1.0使用Knative,后面改用百度函数计算产品
    \end{tablenotes}
\end{threeparttable}
\caption{一些Serverless平台的基础设施}
\label{table:serverless-infras}
\end{table}
\subsection{开发语言}
\subsection{架构设计}